\documentclass[a5paper, 9pt]{memoir}
\usepackage[layout=a5paper,
            ignoreall,
            nomarginpar,
            centering,
            margin=5mm,
            %showframe=true,
            %showcrop=true,
            ]{geometry}
\usepackage{tikz}
\usepackage{xifthen}
\usepackage{xstring}
\setlength{\parindent}{0pt}

\usepackage[T1]{fontenc} % Ligaturen, richtige Umlaute im PDF
\usepackage[utf8]{inputenc} % UTF8-Kodierung für Umlaute usw

\pgfmathsetseed{\number\pdfrandomseed}

\newcommand{\randomwords}{%
Ist verwandt mit der Braut/dem Bräutigam;
Ist das älteste/jüngste Geschwisterkind;
Ist mehr als 50km angereist;
Hat einen ausländischen Pass;
War schon auf mehr als 3 Kontinenten;
Besitzt 3 oder mehr Kraftfahrzeuge;
Hat einen Titel;
Ist mehr als 25 Jahre verheiratet;
War bei der Bundeswehr;
Kann auf mindestens 3 Fremdsprachen bis 10 zählen;
Hat einen Hund;
Hat eine Katze;
Kann Steuerfragen beantworten;
Kennt sich mit Mäusen aus;
Weiß was bei einer Geburt zu tun ist;
Hat im letzten Jahr ein Kind bekommen;
Kennt Braut/Bräutigam seit der Grundschule;
Hat eine Ziege zur Hochzeit verschenkt;
Hat diesen Monat Geburtstag;
Ist Linkshänder;
Hat das Brautpaar zur eigenen Hochzeit eingeladen;
Arbeitet mit Braut/Bräutigam;
Kann ein Instrument spielen;
Kommt aus einem Ort mit weniger als 15000 Einwohnern;
Kommt aus einer Stadt mit mehr als 1 000 000 Einwohnern;
Hat ein Tattoo;
Trinkt lieber Weißwein als Rotwein;
Hat Abschlüsse von mehreren Hochschulen;
Ist kurzsichtig;
Ist weitsichtig;
Hat am gleichen Tag Geburtstag wie Braut/Bräutigam;
Fährt einen Oldtimer;
Besitzt ein E-Bike;
Hat Schuhe zum Wechseln dabei;
Hat eine längere Zeit im Ausland verbracht;
Kennt mindestens 3 IKEA Möbelstücke mit Namen;
Hat seine Steuererklärung noch nicht abgegeben;
Hat mehr als 10 Pflanzen in der Wohnung;
Weiß was Numismatik ist;
%Hat Schuhgröße <36/38/40/42/44/>46;
Ist verlobt;
Hat ein iPhone;
Hat ein Android;
Hat einen Bootsführerschein;
Hat seine Antwortkarte nicht rechtzeitig abgeschickt;
Hat eine olympische Medaille;
Kennt mehr als eine Programmiersprache;
Hat den gleichen Namen wie eine andere Person auf der Hochzeit;
	}
\newcommand{\freifeld}{Hat heute geheiratet}

\pgfmathsetmacro{\cellsize}{2.7}
\pgfmathtruncatemacro{\gridsize}{5}

\pgfmathtruncatemacro{\fieldcount}{\gridsize*\gridsize-1}
\pgfmathtruncatemacro{\bingo}{\fieldcount/2}
\StrCount{\randomwords}{;}[\numwords]

\begin{document}
\pagestyle{empty}
\includegraphics{header.pdf}
%{\Huge\textbf{Hochzeitsbingo}}\hfill {\large Name:\hspace{5cm}
%\printinunitsof{cm}\prntlen{\textwidth}

\vspace{3mm}

Damit ihr euch alle ein bisschen besser kennenlernt haben wir einen Gesprächsstarter vorbereitet.
Die meisten der unten gegebenen Beschreibungen treffen auf einen oder mehrere unserer Gäste zu.
Unterhaltet euch also mit möglichst vielen Leuten.
Schreibt den Namen der entsprechenden Person in das Feld.
Jede Person darf nur in einem Feld auftauchen.
5 richtige Namen in einer Reihe ergeben ein Bingo.
Unter allen, die ein Bingo haben verlosen wir heute Abend eine kleine Überraschung.
Vergesst dafür nicht euren Namen auf den Zettel zu schreiben.
}

\vfill
    \begin{minipage}{\textwidth}%
    \centering%
	\begin{tikzpicture}[every node/.style={align=center, text width=2.4cm}]
	\foreach \f in {0,...,\fieldcount}
	{ \pgfmathtruncatemacro{\x}{mod(\f,\gridsize)}
		\pgfmathtruncatemacro{\y}{div(\f,\gridsize)}
		\draw ({\x*\cellsize},{\y*\cellsize}) rectangle ({(\x+1)*\cellsize},{(\y+1)*\cellsize});
		\ifthenelse{\f=\bingo}
		{   \node at ({(\x+0.5)*\cellsize},{(\y+0.5)*\cellsize}) {\freifeld\\\vfill};
		}
		{   \pgfmathtruncatemacro{\maxvalue}{\numwords-1-\f)}
			\pgfmathtruncatemacro{\myrandom}{random(\maxvalue)}
			\pgfmathtruncatemacro{\mynextrandom}{\myrandom+1}
			\StrBetween[\myrandom,\mynextrandom]{\randomwords}{;}{;}[\randomword]
			\StrDel{\randomwords}{\randomword;}[\randomwords]
			\xdef\randomwords{\randomwords}
			%\node at ({(\x+0.5)*\cellsize},{(\y+0.5)*\cellsize}) {\myrandom-\maxvalue-\randomword};
			\node at ({(\x+0.5)*\cellsize},{(\y+0.5)*\cellsize}) {\randomword};
		}
	}
	\end{tikzpicture}%
	\end{minipage}
\end{document}
